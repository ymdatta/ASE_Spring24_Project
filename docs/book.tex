\documentclass[twocolumn,landscape]{book}
\usepackage{minted}
\usepackage{inconsolata}
\usepackage{amssymb}
\usepackage{pifont}
\usepackage{array}
  \usepackage[T1]{fontenc}
    \usepackage{textcomp}
   \usepackage{mathpazo}
\usepackage{tikz}

\newcommand{\BALL}[1]{\tikz[baseline=(myanchor.base)] \node[circle,fill=.,inner sep=1pt] (myanchor) {\color{-.}\bfseries\footnotesize #1};}


\usepackage{enumitem}
\usepackage[hidelinks]{hyperref}
\setitemize{topsep=0pt,parsep=0pt,partopsep=0pt}
\usepackage[margin=.7in]{geometry}
\usepackage{pdflscape}
 \setlength{\columnsep}{30pt}

\usepackage[utf8]{inputenc}
\usepackage[english]{babel}
\usepackage{natbib}


\usemintedstyle{tango}
\setminted{fontsize=\scriptsize,firstnumber=last,mathescape,frame=lines,linenos,framerule=1pt,rulecolor=\color{gray!40}}

\title{A Little Less AI (in under 500 lines)}
%\subtitle{Multi-objective Sequential Model Optimization}
\author{{\bf Tim Menzies}\\ \url{mailto:youremail@example.com}{timm@ieee.org} \\ \url{http://yourhomepage.com}{http://timm.fyi}}
\date{\today}

\begin{document}
\maketitle

\chapter{1}
For AI and SE, do you know what you are doing? 
Can you build AI software, succinctly, from just a handful of parts? And can other people
understand, critique and improve?

Let's check. Let's build an explanation system for multi-objective semi-supervised learning (which, in this book, is a long-winded
way of saying 
The task explored here is finding effective human-readable rules for multi-goal problems, after labelling just a handful of examples.
\begin{itemize}
  \item By {\em effecive} I mean that the rules can actually select for important parts of the problem space.
  \item And I want to do this by {\em labelling just a few examples}, I mean that dependent and independent variables are different
    Depednent varables are goals like "wealth" which we want to maximize or ``weight'' which we want t minimize..
    Whiele we see indendepntn values all aroudn us, the labels saying that something is ``good'' or ``bad'' are far 
    harder to obtain)..
\end{itemize}
$N$ things can be 
Too many choices, not enough time to look at them all.
\begin{itemize}
\item e.g. Hundreds of cars in a car yard, you try three, then buy one;
\item e.g. You can't test everything -- so you just test a few;
\item e.g. Software has $10^9$ of options -- but you  have time to try a few.
\end{itemize}
So lets apply {\em sequential model optimization}:
\begin{itemize}
\item \citet{xia2020sequential}, 
\citet{hutter2011sequential},
\citet{nair2018finding},
\citet{hsu2018arrow},
\citet{mockus1989bayesian},
\citet{golovin17}



\item e.g. Hundreds of cars in a car yard, you try three, then buy one;
Some terminology
\[ \underbrace{y_1,y_2,...}_{\mathit{}dependent\;variables,\;goals} = f(\underbrace{x_1, x_2, x_3, x_4, x_5, x_6,....}_{\mathit{indendepent variables}})
\]
\end{itemize}
\begin{itemize}
\item 
\item But depndent variables are more expesnive to collect 
\item e.g. A supermarket has 100 apples. Which ones are tasty?
So lets walk data incrementally:
\end{itemize}


\section{Problem}
Too many choices, not enough time to look at them all.
line \ref{one}.
\begin{itemize} 
\item adas
\end{itemize}
\clearpage
\begin{minted}{lua}
  function NUM.new(i,at,txt) -- --> NUM;  constructor; $\label{one}$
  i.at, i.txt = at or 0, txt or "" -- column position 
  i.n, i.mu, i.m2 = 0, 0, 0
  i.lo, i.hi = math.huge, -math.huge 
  i.w = i.txt:find"-$" and -1 or 1 end

function NUM.new(i,at,txt) --> NUM;  constructor; 
  i.at, i.txt = at or 0, txt or "" -- column position 
  i.n, i.mu, i.m2 = 0, 0, 0
  i.lo, i.hi = math.huge, -math.huge 
  i.w = i.txt:find"-$" and -1 or 1 end



function NUM.new(i,at,txt) --> NUM;  constructor; 
  i.at, i.txt = at or 0, txt or "" -- column position 
  i.n, i.mu, i.m2 = 0, 0, 0
  i.lo, i.hi = math.huge, -math.huge 
  i.w = i.txt:find"-$" and -1 or 1 end


function NUM.new(i,at,txt) --> NUM;  constructor; 
  i.at, i.txt = at or 0, txt or "" -- column position 
  i.n, i.mu, i.m2 = 0, 0, 0
  i.lo, i.hi = math.huge, -math.huge 
  i.w = i.txt:find"-$" and -1 or 1 end


function NUM.new(i,at,txt) --> NUM;  constructor; 
  i.at, i.txt = at or 0, txt or "" -- column position 
  i.n, i.mu, i.m2 = 0, 0, 0
  i.lo, i.hi = math.huge, -math.huge 
  i.w = i.txt:find"-$" and -1 or 1 end


function NUM.new(i,at,txt) --> NUM;  constructor; 
  i.at, i.txt = at or 0, txt or "" -- column position 
  i.n, i.mu, i.m2 = 0, 0, 0
  i.lo, i.hi = math.huge, -math.huge 
  i.w = i.txt:find"-$" and -1 or 1 end

function NUM.new(i,at,txt) --> NUM;  constructor; 
  i.at, i.txt = at or 0, txt or "" -- column position 
  i.n, i.mu, i.m2 = 0, 0, 0
  i.lo, i.hi = math.huge, -math.huge 
  i.w = i.txt:find"-$" and -1 or 1 end

function NUM.new(i,at,txt) --> NUM;  constructor; 
  i.at, i.txt = at or 0, txt or "" -- column position 
  i.n, i.mu, i.m2 = 0, 0, 0
  i.lo, i.hi = math.huge, -math.huge 
  i.w = i.txt:find"-$" and -1 or 1 end

function NUM.new(i,at,txt) --> NUM;  constructor; 
  i.at, i.txt = at or 0, txt or "" -- column position 
  i.n, i.mu, i.m2 = 0, 0, 0
  i.lo, i.hi = math.huge, -math.huge 
  i.w = i.txt:find"-$" and -1 or 1 end


\end{minted}



\section{Test section one}

\section{What is Prolog?}
\begin{itemize}
    \item A programming language associated with artificial intelligence and computational linguistics.
    \item Based on formal logic.
    \item Declarative: Describe the problem, not how to solve it.
    \item Known for its ability to handle symbolic reasoning and knowledge representation.
\end{itemize}


   some text here some text here some text here some text here some text here
    \begin{center}
       asdsa
     \end{center}

\section{References}
         \bibliographystyle{natbib}
         \bibliography{slides.bib}

\end{document}
